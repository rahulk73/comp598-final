\def\year{2022}\relax
%File: formatting-instructions-latex-2022.tex
%release 2022.1
\documentclass[letterpaper]{article} % DO NOT CHANGE THIS
\usepackage{aaai22}  % DO NOT CHANGE THIS
\usepackage{times}  % DO NOT CHANGE THIS
\usepackage{helvet}  % DO NOT CHANGE THIS
\usepackage{courier}  % DO NOT CHANGE THIS
\usepackage[hyphens]{url}  % DO NOT CHANGE THIS
\usepackage{graphicx} % DO NOT CHANGE THIS
\urlstyle{rm} % DO NOT CHANGE THIS
\def\UrlFont{\rm}  % DO NOT CHANGE THIS
\usepackage{natbib}  % DO NOT CHANGE THIS AND DO NOT ADD ANY OPTIONS TO IT
\usepackage{caption} % DO NOT CHANGE THIS AND DO NOT ADD ANY OPTIONS TO IT
\DeclareCaptionStyle{ruled}{labelfont=normalfont,labelsep=colon,strut=off} % DO NOT CHANGE THIS
\frenchspacing  % DO NOT CHANGE THIS
\setlength{\pdfpagewidth}{8.5in}  % DO NOT CHANGE THIS
\setlength{\pdfpageheight}{11in}  % DO NOT CHANGE THIS

\usepackage{algorithm}
\usepackage{algorithmic}

\usepackage{newfloat}
\usepackage{listings}
\lstset{%
	basicstyle={\footnotesize\ttfamily},% footnotesize acceptable for monospace
	numbers=left,numberstyle=\footnotesize,xleftmargin=2em,% show line numbers, remove this entire line if you don't want the numbers.
	aboveskip=0pt,belowskip=0pt,%
	showstringspaces=false,tabsize=2,breaklines=true}
\floatstyle{ruled}
\newfloat{listing}{tb}{lst}{}
\floatname{listing}{Listing}
\pdfinfo{
/Title (AAAI Press Formatting Instructions for Authors Using LaTeX -- A Guide)
/Author (AAAI Press Staff, Pater Patel Schneider, Sunil Issar, J. Scott Penberthy, George Ferguson, Hans Guesgen, Francisco Cruz, Marc Pujol-Gonzalez)
/TemplateVersion (2022.1)
}

\setcounter{secnumdepth}{0} %May be changed to 1 or 2 if section numbers are desired.

\title{COVID-19 on Twitter: Heated Discussions}
\author{Antoine Bonnet, \\
  260928321 \\
antoine.bonnet@mail.mcgill.ca,
\\[3ex]
Rahul Kumar,\\
260847297\\
rahul.kumar2@mail.mcgill.ca, 
\\[3ex]
Fabrizzio Sabelli,\\
???\\
fabrizzio.sabelli@mail.mcgill.ca}

\begin{document}

\maketitle

\section{Introduction}
boop

\section{Data}
\noindent The below list of 27 COVID-19 related keywords was written to collect Tweets containing at least one such keyword. This list contains multiple derivations of the COVID-19 name, principal vaccine manufacturers, as well as words related to the pandemic and sanitary measures. 

%\hspace*{-\sectionwidth}
\makebox[0pt][l]{\begin{tabular}[t]{ @{} l }
  COVID \\ COVID-19 \\ COVID19 \\ coronavirus \\ pandemic \\ quarantine \\ screening \\ distancing \\immunity \\immunization
\end{tabular}}\hfill
\begin{tabular}[t]{ l }
  symptoms \\ symptomatic \\ epidemic \\ vax \\ vaxx \\ vaxxed \\ vaccine \\ vaccination \\injection \\ dose 
\end{tabular}\hfill
\makebox[0pt][r]{\begin{tabular}[t]{ l @{} }
  AstraZeneca \\ BioNTech \\ Johnson \& Johnson \\ Moderna \\ Pfizer \\ Sanofi \\ Sinopharm
\end{tabular}}


\noindent From these keywords, 1000 different Tweets were collected on December 6th, 2021. This dataset was then annotated manually by the team. Each Tweet was first given a topic among the list: 
\begin{enumerate}
    \item \textbf{Vaccination}: includes discussion on vaccine manufacturers and vaccination mandates.
    \item \textbf{Variant}: includes discussion around new variants (e.g. Omicron) and their effect on governmental measures.  
    \item \textbf{Sanitary measures}: includes discussion on all governmental sanitary measures like lockdown, quarantining, social restrictions and masks. 
    \item \textbf{Politics and economy}: includes discussion on politics around the world related to COVID-19, as well as the jobs market and economic repercussions of the pandemic.
    \item \textbf{Symptoms and Testing}: includes discussion on COVID-19 symptoms and PCR testing.
    \item \textbf{Pandemic}: includes discussion on new cases, outbreaks, deaths. 
    \item \textbf{Entertainment and Community}: includes discussion around the effect of the pandemic on entertainment, sports and the social community at large. 
    \item \textbf{Unrelated}: all Tweets unrelated to COVID-19 that were picked up by accident. 
\end{enumerate}

\noindent During the annotation process, we realized that a few Tweets (approximately 2\%) had been incorrectly picked up by a keyword and were unrelated to the COVID-19 pandemic. In order to exclude this faulty data, we created an extra category of unrelated Tweets. 


\noindent We also realized that evaluating the sentiment of a Twitter user leads to various ambiguities. For example, a Tweet might be phrased very negatively in opposition to anti-vaccine supporters. In this case, the Tweet would have a negative sentiment, but it would express a positive position towards vaccination. To solve this problem, we evaluated the sentiment of the user only in relation to its approval or disapproval of sanitary measures and vaccination. A value of -1 was given to users expressing disagreement over sanitary measures (e.g. lockdowns, vaccination, masks, quarantining). A value of +1 was given to users agreeing with the implementation of these measures. Finally, a value of 0 was given to Tweets which were either unrelated to COVID-19, or were completely factual (originating from news articles like case counts) or which lacked a definite position on the subject. 

\section{Methods}
boop


\section{Results}
boop

\section{Discussion}
boop

\section{Group Member Contributions}
boop

\section{References}
boop
\end{document}